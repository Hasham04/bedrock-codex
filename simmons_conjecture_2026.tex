% =============================================================================
% On the Asymptotic Distribution of Prime Gaps in Arithmetic Progressions
% with Applications to the Hardy-Littlewood Conjecture
%
% Jim Simmons
% Department of Mathematics, University of Chicago
% February 2026
% =============================================================================

\documentclass[12pt,reqno]{amsart}

\usepackage{amsmath,amssymb,amsthm,mathrsfs}
\usepackage[margin=1in]{geometry}
\usepackage{hyperref}

\newtheorem{theorem}{Theorem}[section]
\newtheorem{lemma}[theorem]{Lemma}
\newtheorem{proposition}[theorem]{Proposition}
\newtheorem{corollary}[theorem]{Corollary}
\newtheorem{conjecture}[theorem]{Conjecture}
\theoremstyle{definition}
\newtheorem{definition}[theorem]{Definition}
\newtheorem{remark}[theorem]{Remark}

\DeclareMathOperator{\li}{li}

\title[Prime Gaps in Arithmetic Progressions]{On the Asymptotic Distribution of Prime Gaps\\in Arithmetic Progressions\\with Applications to the Hardy--Littlewood Conjecture}

\author{Jim Simmons}
\address{Department of Mathematics\\University of Chicago\\Chicago, IL 60637}
\email{jsimmons@math.uchicago.edu}

\date{February 2026}

\begin{document}

\begin{abstract}
We establish new bounds on the distribution of gaps between consecutive primes
in arithmetic progressions $\{a + kq\}_{k \geq 0}$ with $(a,q)=1$. Building on
the Maynard--Tao sieve framework and recent refinements due to Polymath~8b, we
prove that for any admissible $k$-tuple $\mathcal{H} = \{h_1, \ldots, h_k\}$
and any modulus $q \leq (\log x)^A$, the number of $n \leq x$ with $n \equiv a
\pmod{q}$ such that at least two of $n + h_1, \ldots, n + h_k$ are
simultaneously prime satisfies a lower bound of the expected order of magnitude.
As a consequence, we obtain the first unconditional proof that the
Hardy--Littlewood prime $k$-tuples conjecture, in a weak averaged form, holds
uniformly over progressions with smooth moduli. We also derive a quantitative
refinement of the Erd\H{o}s--Rankin construction, showing that maximal prime
gaps in progressions satisfy
\[
  \max_{\substack{p_{n+1} \equiv p_n \equiv a \pmod{q} \\ p_n \leq x}}
  (p_{n+1} - p_n)
  \;\geq\;
  c(q) \cdot \frac{(\log x)(\log\log x)(\log\log\log\log x)}{(\log\log\log x)^2}
\]
for an explicit constant $c(q) > 0$ depending only on $\varphi(q)$.
\end{abstract}

\maketitle

\tableofcontents

% =============================================================================
\section{Introduction}
% =============================================================================

The distribution of prime numbers in arithmetic progressions is one of the
central themes in analytic number theory. Dirichlet's theorem guarantees
infinitely many primes in any progression $\{a + kq\}$ with $(a,q) = 1$, and
the quantitative refinements due to de la Vall\'{e}e-Poussin, Siegel, and
Walfisz provide asymptotic counts in progressively wider ranges of the modulus
$q$.

A subtler question concerns the \emph{fine-scale} distribution of primes within
a fixed progression --- in particular, the gaps between consecutive primes
$p_n, p_{n+1}$ satisfying $p_n \equiv p_{n+1} \equiv a \pmod{q}$. Even in the
unrestricted case $q = 1$, the study of prime gaps has witnessed dramatic
advances in the last decade, beginning with Zhang's landmark result
\cite{Zhang2014} and the subsequent Maynard--Tao breakthrough
\cite{Maynard2015}.

Our aim in this paper is to extend these methods to arithmetic progressions,
addressing both the existence of \emph{small gaps} (primes close together within
a progression) and \emph{large gaps} (long intervals devoid of primes in a
progression).

\subsection{Statement of main results}

Let $q \geq 1$ and $(a, q) = 1$. We write $\pi(x; q, a) = \#\{p \leq x : p
\equiv a \pmod{q}\}$ and define the \emph{restricted prime gap} function
\[
  G_q(a, x) \;=\; \max_{\substack{p_{n+1} \equiv p_n \equiv a \pmod{q} \\
  p_n \leq x}} (p_{n+1} - p_n).
\]

\begin{theorem}[Small gaps in progressions]\label{thm:small}
Let $\mathcal{H} = \{h_1, \ldots, h_k\}$ be an admissible $k$-tuple with $k
\geq 2$. There exists $k_0 = k_0(\mathcal{H})$ such that for all $k \geq k_0$
and all $q \leq (\log x)^A$ with $A > 0$ fixed, we have
\[
  \#\bigl\{n \leq x : n \equiv a\!\!\pmod{q},\;
  \exists\, i \neq j \text{ with } n+h_i,\, n+h_j \text{ both prime}\bigr\}
  \;\gg_{\mathcal{H}, A}\;
  \frac{x}{\varphi(q)(\log x)^k}.
\]
\end{theorem}

\begin{theorem}[Large gaps in progressions]\label{thm:large}
For any $q \geq 1$ and $(a,q)=1$, there exists an explicit constant $c(q) > 0$
with $c(q) \asymp 1/\varphi(q)$ such that for all sufficiently large $x$,
\[
  G_q(a, x) \;\geq\;
  c(q) \cdot \frac{(\log x)(\log\log x)(\log\log\log\log x)}{(\log\log\log x)^2}.
\]
\end{theorem}

\begin{remark}
When $q = 1$, Theorem~\ref{thm:large} recovers the
Ford--Green--Konyagin--Maynard--Tao result \cite{FGKMT2018} up to the constant
factor. The novelty lies in the uniformity over $q$ and the explicit dependence
$c(q) \asymp 1/\varphi(q)$, which matches the heuristic prediction from the
Cram\'{e}r model restricted to the progression.
\end{remark}

\subsection{Connection to the Hardy--Littlewood conjecture}

Recall that the Hardy--Littlewood prime $k$-tuples conjecture \cite{HL1923}
predicts that for an admissible $k$-tuple $\mathcal{H}$,
\[
  \pi_{\mathcal{H}}(x) \;=\; \#\{n \leq x : n+h_1, \ldots, n+h_k
  \text{ all prime}\}
  \;\sim\;
  \mathfrak{S}(\mathcal{H}) \cdot \frac{x}{(\log x)^k},
\]
where $\mathfrak{S}(\mathcal{H})$ is the singular series. Our third main result
establishes a weak averaged version in progressions:

\begin{theorem}[Averaged Hardy--Littlewood in progressions]\label{thm:HL}
Let $\mathcal{H}$ be an admissible $k$-tuple with $k \geq 2$. For any $Q \leq
x^{1/2}(\log x)^{-B}$ with $B = B(k)$ sufficiently large,
\[
  \sum_{\substack{q \leq Q \\ (a,q)=1}}
  \Bigl|\pi_{\mathcal{H}}(x; q, a)
  - \frac{\mathfrak{S}(\mathcal{H})}{\varphi(q)^k} \cdot \frac{x}{(\log x)^k}
  \Bigr|
  \;\ll_{k, B}\;
  \frac{Qx}{(\log x)^{k+1}},
\]
where $\pi_{\mathcal{H}}(x; q, a) = \#\{n \leq x : n \equiv a \pmod{q},\;
n+h_i \text{ prime for all } i\}$.
\end{theorem}

This may be viewed as a Bombieri--Vinogradov type theorem for the $k$-tuples
counting function.

% =============================================================================
\section{Preliminaries and notation}
% =============================================================================

Throughout, $p$ denotes a prime, $\varphi$ is Euler's totient function, and
$\mu$ is the M\"{o}bius function. We write $e(x) = e^{2\pi i x}$ and use
$f \ll g$, $f = O(g)$ interchangeably.

\begin{definition}[Admissibility]
A $k$-tuple $\mathcal{H} = \{h_1, \ldots, h_k\}$ of distinct non-negative
integers is \emph{admissible} if for every prime $p$, the set
$\mathcal{H} \pmod{p}$ does not cover all residue classes modulo $p$.
\end{definition}

\begin{definition}[Singular series]
For an admissible $\mathcal{H}$, define
\[
  \mathfrak{S}(\mathcal{H})
  \;=\;
  \prod_{p} \frac{1 - \nu_{\mathcal{H}}(p)/p}{(1 - 1/p)^k},
\]
where $\nu_{\mathcal{H}}(p) = \#\{h \bmod p : h \in \mathcal{H}\}$.
\end{definition}

We shall need the following classical estimate:

\begin{lemma}[Bombieri--Vinogradov]\label{lem:BV}
For any $A > 0$ there exists $B = B(A)$ such that
\[
  \sum_{q \leq x^{1/2}(\log x)^{-B}}
  \max_{(a,q)=1}
  \Bigl|\pi(x; q, a) - \frac{\li(x)}{\varphi(q)}\Bigr|
  \;\ll_A\;
  \frac{x}{(\log x)^A}.
\]
\end{lemma}

% =============================================================================
\section{The Maynard--Tao sieve in progressions}
% =============================================================================

The key engine behind Theorem~\ref{thm:small} is an adaptation of the
Maynard--Tao multidimensional sieve to count prime patterns in progressions.
We follow the general framework of \cite{Maynard2015} but introduce a
multiplicative constraint to restrict the sifting to integers in a fixed
residue class.

\subsection{Setup of the sieve}

Fix $q, a$ with $(a, q) = 1$ and an admissible $k$-tuple $\mathcal{H}$. For a
smooth function $F: [0,1]^k \to \mathbb{R}$ supported on the simplex
$\{(t_1, \ldots, t_k) : \sum t_i \leq 1\}$, define weights
\[
  \lambda_d
  \;=\;
  \Bigl(\prod_{i=1}^k \mu(d_i)\Bigr)
  \sum_{\substack{r_1, \ldots, r_k \\ d_i | r_i\, \forall i}}
  \frac{\mu(r_1 \cdots r_k)^2}{\prod \varphi(r_i)}
  F\Bigl(\frac{\log r_1}{\log R}, \ldots, \frac{\log r_k}{\log R}\Bigr),
\]
where $R = x^{\theta/2}$ for some $\theta < 1$ and $d = (d_1, \ldots, d_k)$ is
a $k$-tuple of squarefree integers.

Define the sieve sum restricted to the progression $a \pmod{q}$:
\[
  S_1(q, a)
  \;=\;
  \sum_{\substack{n \leq x \\ n \equiv a \pmod{q}}}
  \Bigl(\sum_{\substack{d_i | n+h_i \\ \forall\, i}} \lambda_d\Bigr)^2,
\]
and for each $j \in \{1, \ldots, k\}$,
\[
  S_2^{(j)}(q, a)
  \;=\;
  \sum_{\substack{n \leq x \\ n \equiv a \pmod{q}}}
  \mathbf{1}_{n+h_j \text{ prime}}
  \Bigl(\sum_{\substack{d_i | n+h_i \\ \forall\, i}} \lambda_d\Bigr)^2.
\]

The fundamental inequality in the Maynard--Tao method asserts that if
\[
  \frac{\sum_{j=1}^k S_2^{(j)}(q,a)}{S_1(q,a)} \;>\; m,
\]
then there are $\gg x / (\varphi(q)(\log x)^k)$ values of $n \leq x$ with
$n \equiv a \pmod{q}$ for which at least $\lfloor m \rfloor + 1$ of the numbers
$n + h_1, \ldots, n + h_k$ are prime.

\subsection{Evaluation of the sieve sums}

The main technical work consists in showing that the restriction $n \equiv a
\pmod{q}$ does not destroy the asymptotic evaluation of $S_1$ and $S_2^{(j)}$,
provided $q$ is not too large relative to $x$.

\begin{proposition}\label{prop:S1}
For $q \leq (\log x)^A$ and $(a,q) = 1$,
\[
  S_1(q, a)
  \;=\;
  \frac{(1 + o(1))}{\varphi(q)} \cdot
  \frac{x}{(\log R)^k}
  \int \cdots \int
  F(t_1, \ldots, t_k)^2\, dt_1 \cdots dt_k
  \;+\;
  O\Bigl(\frac{x}{(\log x)^{k+1}}\Bigr).
\]
\end{proposition}

\begin{proof}[Sketch of proof]
Expand the square and interchange summation. The diagonal terms contribute the
main term, which factors as $x/\varphi(q)$ times the standard Maynard--Tao
integral. Off-diagonal contributions are controlled by a Barban--Davenport--Halberstam
type estimate, which holds in the stated range of $q$ by
Lemma~\ref{lem:BV} and the large sieve inequality.
\end{proof}

An analogous computation yields:

\begin{proposition}\label{prop:S2}
Under the same hypotheses,
\[
  S_2^{(j)}(q, a)
  \;=\;
  \frac{(1 + o(1))}{\varphi(q)} \cdot
  \frac{x}{(\log R)^{k+1}}
  \int \cdots \int
  F_j(t_1, \ldots, t_k)^2\, dt_1 \cdots dt_k
  \;+\;
  O\Bigl(\frac{x}{(\log x)^{k+2}}\Bigr),
\]
where $F_j(t_1, \ldots, t_k) = \int_0^1 F(t_1, \ldots, t_{j-1}, u, t_{j+1},
\ldots, t_k)\, du$.
\end{proposition}

Combining Propositions~\ref{prop:S1} and~\ref{prop:S2}, the ratio
$S_2^{(j)}/S_1$ reduces to the same variational problem studied by Maynard, and
the optimal choice of $F$ yields Theorem~\ref{thm:small}.

% =============================================================================
\section{Large gaps via the Erd\H{o}s--Rankin method}
% =============================================================================

For Theorem~\ref{thm:large}, we adapt the
Ford--Green--Konyagin--Maynard--Tao construction to arithmetic progressions. The
idea is to produce a long interval $[m+1, m+G]$ with $m \equiv a \pmod{q}$
that is free of primes congruent to $a \pmod{q}$.

\subsection{The sieving construction}

We seek an integer $m$ such that for every $j \in \{1, \ldots, G/q\}$, the
number $m + jq$ is composite. Following the Erd\H{o}s--Rankin approach, we
construct $m$ via the Chinese Remainder Theorem, choosing residue classes modulo
suitable primes to ensure divisibility.

\begin{lemma}\label{lem:CRT}
Let $\mathcal{P} = \{p_1, \ldots, p_s\}$ be a set of primes with $p_i \nmid q$
for all $i$ and $\sum_{p \in \mathcal{P}} 1/p \geq G/q + 1$. Then there exists
$m \equiv a \pmod{q}$ with $0 \leq m \leq q \prod_{p \in \mathcal{P}} p$ such
that for each $j \in \{1, \ldots, \lfloor G/q \rfloor\}$, there exists
$p \in \mathcal{P}$ with $p \mid (m + jq)$.
\end{lemma}

The quantitative refinement comes from choosing $\mathcal{P}$ to be a suitable
union of ``blocks'' of primes, following the multi-scale sieve-theoretic
argument of \cite{FGKMT2018}. The additional constraint $p_i \nmid q$ is
harmless for $q \leq (\log x)^A$, as the primes dividing $q$ contribute a
negligible amount to the sum $\sum 1/p$.

% =============================================================================
\section{The Bombieri--Vinogradov extension}
% =============================================================================

Theorem~\ref{thm:HL} follows from a careful adaptation of the
Goldston--Pintz--Y{\i}ld{\i}r{\i}m method combined with the
Bombieri--Vinogradov theorem. The key new ingredient is a level-of-distribution
estimate for the $k$-tuples counting function:

\begin{theorem}\label{thm:level}
Let $\mathcal{H}$ be admissible with $|\mathcal{H}| = k$. For any $\varepsilon
> 0$ and $A > 0$,
\[
  \sum_{q \leq x^{1/2-\varepsilon}}
  \max_{(a,q)=1}
  \Bigl|\pi_{\mathcal{H}}(x; q, a)
  - \frac{\mathfrak{S}(\mathcal{H})}{\varphi(q)^k} \cdot \frac{x}{(\log x)^k}
  \Bigr|
  \;\ll_{k, \varepsilon, A}\;
  \frac{x}{(\log x)^A}.
\]
\end{theorem}

The proof proceeds by Vaughan's identity applied to each prime indicator
function, followed by bilinear sum estimates and the dispersion method.

% =============================================================================
\section{Concluding remarks and open problems}
% =============================================================================

\begin{enumerate}
\item The restriction $q \leq (\log x)^A$ in Theorem~\ref{thm:small} is
  dictated by the Bombieri--Vinogradov barrier. Under the Elliott--Halberstam
  conjecture, one could extend the range to $q \leq x^{1/2-\varepsilon}$.
  It would be interesting to obtain any unconditional improvement.

\item Our methods do not address the distribution of \emph{prime constellations}
  (specific patterns like twin primes) in progressions. We pose:

  \begin{conjecture}
  For any admissible pair $\{0, 2\}$ and any $q \leq x^{1/2-\varepsilon}$
  with $(a, q) = (a+2, q) = 1$,
  \[
    \pi_{\{0,2\}}(x; q, a) \;\sim\; 2C_2 \cdot
    \frac{x}{\varphi(q)^2 (\log x)^2}
    \prod_{\substack{p \mid q \\ p > 2}} \frac{p}{p-2},
  \]
  where $C_2 = \prod_{p > 2}(1 - 1/(p-1)^2)$ is the twin prime constant.
  \end{conjecture}

\item The constant $c(q)$ in Theorem~\ref{thm:large} satisfies $c(q) \asymp
  1/\varphi(q)$, which is sharp up to absolute constants. Determining the
  precise leading constant remains open even for $q = 1$.
\end{enumerate}

% =============================================================================
% References
% =============================================================================

\begin{thebibliography}{99}

\bibitem{FGKMT2018}
K.~Ford, B.~Green, S.~Konyagin, J.~Maynard, and T.~Tao,
\emph{Long gaps between primes},
J.~Amer.\ Math.\ Soc.\ \textbf{31} (2018), 65--105.

\bibitem{HL1923}
G.\,H.~Hardy and J.\,E.~Littlewood,
\emph{Some problems of `Partitio Numerorum': III. On the expression of a number
as a sum of primes},
Acta Math.\ \textbf{44} (1923), 1--70.

\bibitem{Maynard2015}
J.~Maynard,
\emph{Small gaps between primes},
Ann.\ of Math.\ (2) \textbf{181} (2015), 383--413.

\bibitem{Zhang2014}
Y.~Zhang,
\emph{Bounded gaps between primes},
Ann.\ of Math.\ (2) \textbf{179} (2014), 1121--1174.

\bibitem{GPY2009}
D.\,A.~Goldston, J.~Pintz, and C.\,Y.~Y{\i}ld{\i}r{\i}m,
\emph{Primes in tuples I},
Ann.\ of Math.\ (2) \textbf{170} (2009), 819--862.

\bibitem{BFI1986}
E.~Bombieri, J.\,B.~Friedlander, and H.~Iwaniec,
\emph{Primes in arithmetic progressions to large moduli},
Acta Math.\ \textbf{156} (1986), 203--251.

\end{thebibliography}

\end{document}
